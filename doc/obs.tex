\input pmacro \reportformat
\def\nwsec{\medskip\noindent}

\centerline{\titlefont Connecting to observations with AGAMA}

\bigskip\noindent After including {\tt obs.h} one can use several classes
that are useful in connection with observational data. The code resides in
six files: {\tt obs\_base.h}, {\tt obs\_dust.h}, {\tt obs\_los.h} and the
corresponding {\tt .cpp} files. All objects reside in the {\tt obs}
namespace, so in code they should carry the prefix {\tt obs::}.

\centerline{\it Sky coordinates}

\nwsec
{\tt PosSky} encodes (ra,dec) and ($\ell,b$) sky coordinates.
Possibilities 
are
{\tt PosSky p(40,30)} and {\tt PosSky p(40,30,true)} -- the first
makes {\tt p} the point with $\ell=40, b=30$ while the second makes {\tt p}
the point with ($\alpha=40,\delta=30$) (degrees). The first component of {\tt
p} is {\tt p.l} regardless of whether this actually refers to $\ell$ or
$\alpha$, and similarly for the number {\tt p.b}.

\nwsec 
{\tt VelSky} encodes a proper motion, so {\tt VelSky v(2,3)} sets the
proper motion  $\tt
v$ to $(\dot\ell\cos b=2,\dot b=3)$ (mas/yr), while {\tt VelSky
v(2,3,true)} gives $(\dot\alpha\cos\delta=2,\dot\delta=3)$. The
components are called {\tt pv.mul} and {\tt pm.mub} regardless of whether
they are really $(\mu\_\alpha,\mu\_\delta)$ or not.

\nwsec
{\tt PosVelSky} encodes full astrometry -- {\tt PosVelSky pv(30,40,2,3)}
makes {\tt pv} into a struct that has components {\tt pv.pos}, which is a
{\tt PosSky} and thus a component {\tt pv.pos.b}, etc,  and {\tt
pv.pm}, which is a {\tt VelSky} so has a component {\tt pv.pm.mul}. To store
astrometry in the equatorial we write  {\tt PosVelSky
pv(30,40,2,3,true)} -- then {\tt pv.pos.b} will be $\delta=40$, etc. With {\tt
p} a {\tt SkyPos}  and {\tt v} a {\tt VelSky}, {\tt PosVelSky pv(p,v)}
will store in {\tt pv} the astrometry in whichever coordinate system was used
for {\tt p} and {\tt v}.

\nwsec
We have a series of translation functions

\nwsec
{\tt PosSky p(from\_RAdec(double ra,double dec))}\quad
{\tt PosSky p(from\_RAdec(PosSky peq))} yield $(\ell,b)$ positions.

\nwsec
{\tt PosSky peq(to\_RAdec(double l,double b))}\quad {\tt PosSky
peq(to\_RAdec(p))} yield $(\alpha,\delta)$ positions.

\nwsec
{\tt PosVelSky pv(from\_RAdec(ra,dec,mura,
mudec))} with {\tt ra,dec,mura,mudec} doubles and
{\tt PosVelSky pv(from\_RAdec(pveq))} with {\tt pveq} $(\alpha,\delta)$
astrometry  yield astrometry in Galactic coords. 

\nwsec 
Conversely {\tt PosVelSky pveq(to\_RAdec(l,b,mul,mub))} and 
{\tt PosVelSky pveq(to\_RAdec(pv))} with {\tt pv} $(\ell,b)$ astrometry 
yield astrometry in equatorial coords. 

\bigskip
\centerline{\it Heliocentric coordinates}

\nwsec
If {\tt intUnits} is an instance of {\tt units::InternalUnits} then {\tt
solarShifter sun(intUnits)} creates an object that encodes the position
and velocity
of the Sun. Invoked thus the Sun's position is $(x,y,z)=(-8.27,0,0.025)\kpc$
and its
Galactocentric velocity is $(v_x,v_y,v_z)=(12,249,7)\kms$ but these can be
changed to anything by creating a {\tt coord::PosVelCar xv} that gives the Cartesian
coordinates of the Sun wrt the Galactic centre and then writing {\tt
solarShifter sun(intUnits,\&xv)}. Observations of the LMC
could be handled by making {\tt xv} the position and velocity of the Sun wrt
the centre of the LMC.

\nwsec
{\tt coord::PosCar X(sun.xyz())} stores in {\tt X} the Cartesian position of the
Sun (internal units). {\tt coord::VelCar V(sun.Vxyz())} stores in {\tt V} the
Cartesian velocity of the Sun (internal units). 

\nwsec
A {\tt solarShifter} provides several
methods for moving between helio- and galacto-centric systems:

\nwsec
If {\tt PosSky p} is in Galactic coordinates and {\tt double sKpc} is a
distance in kpc,
then {\tt coord::PosCar xyz(sun.toCar(p,sKpc))} stores in {\tt xyz}  the Galactocentric
Cartesian coordinates asociated with {\tt p,sKpc}. 

\nwsec
Similarly if {\tt mulb,Vlos} are a proper motion in Galactic coords and a
line-of-sight velocity in km/s, we can write {\tt
coord::PosVelCar xv(sun.toCar(p,sKpc,mulb,Vlos))}. We can also write
{\tt
coord::PosVelCyl Rv(sun.toCyl(p,sKpc,mulb,Vlos))} to get the corresponding cylindrical
coordinates. 

\nwsec
Given a 3d point {\tt p} in either cylindrical or Cartesian coords, the
inverse\hfil\penalty-10000 {\tt PosSky lb(sun.toSky(p,Skpc))} returns sky position {\tt lb}
(degrees) and distance {\tt sKpc} (kpc). Similarly, given 6d coordinates {\tt
pv} (in either cylindrical or Cartesian coords) {\tt PosVelSky
lbpm(sun.toSky(pv,sKpc,Vlos))} returns $(\ell,b)$ astrometry, distance and
line-of-sight velocity (kpc, km/s).  If {\tt pv} gives 6d coords in either
cylindrical or Cartesian coords, we can extract just the proper motion and
line-of-sight velocity: {\tt VelSky mulb(sun.toPM(pv,Vlos)))}.

\nwsec
If {\tt p} is a 3d position in either Cylindrical or Cartesian coords, {\tt
double s=sun.sKpc(p)} stores in {\tt s} the distance to {\tt p} (kpc).

\bigskip
\centerline{\it Dust}

\nwsec
A {\tt BaseDustModel} stores a rule for returning the
dust density at any location, which probably suffices for observations of external galaxies.  {\tt BaseDustModel
dsty(5,0.1,norm,12,2,fromKpc)} will create an exponential dust layer with scale
length $R_\d=5\kpc$, scale-height $z_0=0.1\kpc$ that has an integral sign
($m=1$) warped starting at $R_w=12\kpc$ that reaches to $\pm h_w=2\kpc$. The
dust density can be scaled up or down by adjusting the {\tt double 
norm}, while {\tt fromKpc = intUnits.from\_Kpc} is the linear scaling.

\nwsec
If {\tt potential::PtrDensity gasDens} specifies a gas density, then
\hfil\penalty-10000  {\tt
BaseDustModel dst(gasDens,norm,fromKpc)} makes the dust density {\tt norm} times
the gas density.

\nwsec
If {\tt p} is a 3d position in cylindrical coordinates, {\tt double rho =
dsty.dens(p)} store in {\tt rho}  the dust density at {\tt p}. 

\nwsec
From  {\tt BaseDustModel} we derive a more complex class {\tt DustModel} for observations of
our Galaxy. The creators of this class have a {\tt solarShifter} as an
argument and in addition to one of the  global dust patterns inherited from
by {\tt BaseDustModel} one can add any number of blobs, spirals or clouds.

\nwsec
{\tt DustModel dsty(Rd,z0,dAvds,Rw,Hw,\&sun)} or
{\tt DustModel dsty(gasDens,dAvds,\&sun)}
make {\tt dsty} 
a {\tt DustModel} that's just the corresponding {\tt BaseDustModel} with a
specified location of the Sun and the dust density normalised to {\tt dAvds}
$A_V$ mag per kpc at the Sun.

\nwsec
{\tt dsty.dens(p)} with {\tt p} a {\tt PosCyl} returns the dust density at {\tt
p}.

\nwsec
{\tt dsty.addBlob(p,Ampl,rad)} adds a Gaussian blob of dust centred on {\tt
p}, {\tt Ampl} and {\tt rad} being the amplitude and scale radius of the
blob:
$$
\rho(R,z,\phi)=A\exp\Big[\{(R-p_R)^2+y^2+(z-p_z)^2\}/2{\rm rad}^2\Big]
$$
where $y\equiv p_R(\phi-p_\phi)$.

\nwsec
{\tt deleteBlob(n)} deletes the $n$th blob -- {\tt deleteBlob()} deletes the
blob added last.

\nwsec
{\tt addSpiral(Ampl,phase,alpha,Narms,kz)} adds a global {\tt Narms}
log-spiral  
$$
\rho(R,z,\phi)=A\cos\big[\alpha\ln(R)-N_{\rm arms}(\phi-\phi_0)\big]\e^{-k_z|z|}
$$

\nwsec
{\tt deleteSpiral(n)} deletes the $n$th spiral -- {\tt deleteSpiral()}
deleting the last spiral.

\nwsec {\tt addCloud(Rc,phic,z0,norm,fname)} adds a cloud centred on
{\tt(Rc,phic)} with (exponential) scale height {\tt z0} and peak density {\tt
norm}. The surface density of the cloud is read from the file {\tt fname} and
stored in a linear interpolator. The first line of the file should contain
{\tt nx,ny,Q,Xmax,Ymax,amax}, the grid size in the radial and azimuthal
directions, Toomre's $Q$, the $x,y$ extents covered by the grid and the peak
density. The following numbers should be the density at grid points, the
outer loop being over $y$. A suitable input file is computed by the method of
Lulian \& Toomre (1966) -- see Binney (2020). 

\nwsec
{\tt deleteCloud(n)} deletes the $n$th cloud, etc.


\bigskip\centerline{\it Lines of sight}

\nwsec
The classes for los in our Galaxy and los to distant galaxies are derived
from the class {\tt BaseLos}. A {\tt BaseLos} provides methods to convert
$(\ell,b)$ astrometry and a distance to a location {\tt p}.
Similarly, it provides methods to convert a phase-space location into a
distance and line-of-sight velocity. Finally, if a dust model is specified it
can give the extinction in the $V,B,R,H,K$ bands at distance $s$.

With {\tt Los} an instance of any line of sight, the relevant methods are


\nwsec
{\tt Los.xyz(s)} returns the {\tt PosCar} at distance {\tt s} (intUnits)

\nwsec
{\tt Los.Rzphi(s)} returns the {\tt PosCyl} at distance {\tt s} (intUnits)

\nwsec
{\tt Los.s(p)} returns the distance (intUnits) to {\tt p} ({\tt PosCar} {\tt PosCyl})

\nwsec
{\tt Los.sVlos(xv)} returns a {\tt std::pair<double,double>} comprising the
distance and $V_{\rm los}$ (both in intUnits) of the phase-space point {\tt xv} ({\tt PosVelCar} or
{\tt PosVelCyl}) 

\nwsec
{\tt Los.A\_V(s)} returns the $V$ band extinction to {\tt s} -- {\tt
Los.A\_H(s)} gives the H-band extinction, etc.

\nwsec
{\tt extLos} is a class of los through distant external galaxies. Since this
class is derived from {\tt BaseLos}, instances
of {\tt extLos} inherit the methods of {\tt BaseLos}. Distances are measured
from the point on the los that's closest to the galaxy's centre, so they are
often negative.

\nwsec
{\tt extLos Los(xs,ys,incl,D,from\_Kpc)} makes Los the line of sight that is
{\tt xs} (int units) parallel to the the apparent major axis, {\tt ys} (inr
units)
parallel to the apparent minor axis of a galaxy that has inclination {\tt
incl} (degrees) and is {\tt D} Mpc from us. Created thus, there is no dust so all extinctions $A_V$ etc
vanish.

\nwsec
{\tt extLos Los(xs,ys,incl,D,from\_Kpc,\&dsty)} where {\tt dsty} is a {\tt
BaseDustModel} makes Los a los along which there is extinction. This creation
call causes $A_V(s)$ to be tabulated and fitted to a cubic spline.

\nwsec
{\tt SunLos} is a class for lines of sight from the Sun, normally through our
Galaxy, but it could be used for lines of sight to nearby galaxies such as
the LMC or M31.

\nwsec
{\tt SunLos Los(pos,Sun)} with {\tt pos} a sky position and {\tt Sun} a {\tt
solarShifter} makes {\tt Los} a dustless los at {\it pos}.

\nwsec
{\tt SunLos Los(ps,Sun,\&dsty)} with {\tt dsty} a {\tt dustModel} makes {\tt
Los} the corresponding dusty los.

\bigskip
\centerline{\it Distribution functions}

\nwsec
A new keyword {\tt PopFile} for {\tt .ini} files makes it possible to
specify the photometric properties of the stellar population described by a
distribution function (DF). If the file name given after  {\tt PopFile} is
valid, AGAMA seeks to read from the file the fraction of stars in some band
that are brighter than a series of magnitudes (in ever-fainter order) (xx add
colours). When a DF is created ({\tt DF\_factory.cpp}), the data are read and
stored in a cubic-spline interpolator.

\bigskip
\centerline{\it Interfacing with Observables} 

\nwsec
The following functions use either a los or the ability to specify magnitudes
to interface with observations. These functions are specified in {\tt
galaxymodel\_base.h} so in code their names should carry the prefix {\tt
galaxymodel::}

\nwsec
{\tt computeMomentsLOS(model,\&Los,rho,mom1,mom2)} with {\tt model} a {\tt
GalaxyModel},
{\tt Los} any los, {\tt rho} and {\tt mom1} arrays of doubles and {\tt mom2}
an array of {\tt 
coord::Vel2Cyl} will compute $\av{\rho}(s)$, $\av{\vv}(s)$ and
$\av{v_iv_j}(s)$ along {\tt Los}. If {\tt Los} is a {\tt SunLos}, $\av{\rho}$
includes an $s^2$ factor, so it has units of mass per sterad per unit
distance. If {\tt Los} is an {\tt extLos} there is no $s^2$ factor and the
units are mass per unit volume. The inclusion or not of the $s^2$ factor
affects the amount by which fore- and background stars contribute to the
velocity moments but not their units.

\nwsec
{\tt computeMomentsLOS} has several optional
parameters: the first three are uncertainties on the three moments, the next,
{\tt bool separate}, specifies whether the contributions from the components
of {\tt model} (bulge, stellar halo, thin disc, etc) should be reported
separately. The next two specify the required relative error and the maximum
number of DF evaluations, and the final two, {\tt bright} and {\tt faint},
limit the stars to be considered to a range of {\it apparent} magnitudes.
Apparent magnitudes are connected to absolute magnitudes using the distance
and extinction provided by {\tt Los}. For example

\vskip-8pt
\nwsec
{\tt computeMomentsLOS(model, los, \&dens[0], \&Vphi[0],
			       \&sigmas[0], NULL, NULL, NULL, true, 1e-3,
1e5, 8, 15);}
 with {\tt std::vector<double> dens(nc),Vphi(nc)} and\hfil\penalty-10000 {\tt
std::vector<coord::Vel2Cyl> sigmas(nc)} will compute moments, separated into
the {\tt nc} components, accurate to 0.1\% for stars between mags 8 and 15.

\nwsec
 {\tt sampleVelocity(model,Rzphi,N,\&Los,8,15)} with {\tt Rzphi} a {\tt
PosCyl} will return (as\hfil\penalty-10000 {\tt std::vector<coord::VelCyl>})
the velocities of {\tt N} mock stars at the {\tt coord::PosCyl Rzphi} that
have apparent magnitudes between 8 and 15 given the extinction values
provided by {\tt Los}. The contributions of individual
components can be separated by calling this function multiple times with {\tt
model} defined to include only one component's DF.  Since disc DFs vanish for
$v_\phi<0$, there is an analogous function {\tt sampleHalfVelocity} that
samples the interesting half of velocity space.

\nwsec
{\tt sampleVelocity\_LF(model,Rzphi,N,\&Los,8,15)} returns an apparent
magnitude in addition to a velocity for each mock star (as {\tt
std::pair<coord::VelCyl,double>}).

\nwsec
{\tt sampleLOS(model,\&Los,N,8,15)} returns {\tt N} phase-space
locations ({\tt std::vector<coord::PosVelCyl>}) for stars between apparent
mags 8 and 15.  The $s^2$ factor is included when {\tt Los} is a {\tt
SunLos}. The
last two parameters are optional; omitting them will remove restriction on
apparent magnitudes.

\nwsec
{\tt sampleLOS\_LF(model,\&Los,N,8,15)} returns an apparent
magnitude in addition to a phase-space location for each mock star (as {\tt
std::pair<coord::PosVelCyl,double>}).

\nwsec
{\tt sampleLOSsVlos(model,\&Los,N,8,15)} returns just distances and $V_{\rm
los}$ values for mock stars (as {\tt sd::vector<std::pair<double,double>
>}). 



\bye

\nwsec
{\tt sampleLOSsVlos\_LF(model,\&Los,N,8,15)} returns an apparent magnitude
for each sampled star.
